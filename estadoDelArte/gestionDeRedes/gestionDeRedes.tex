El crecimiento de la complejidad de la estructura de las organizaciones en los últimos 40 años ha 
conllevado también aumento de la complejidad de sus redes de ordenadores. A medida que ampliamos y 
mejoramos nuestras redes, estas se vuelven más complejas y como resultado más difíciles de 
administrar. Si hace 40 años un administrador podría llevar a cabo su trabajo usando herramientas 
muy simples, con el tiempo estas han demostrado ser obsoletas e insuficientes y por este motivo se 
inició el desarrollo de las tecnologías de gestión de redes necesarias para igualar su nivel de 
sofisticación.

Durante su desarrollo, los investigadores, desarrolladores y usuarios del conjunto de protocolos 
DARPA/DoD TCP/IP experimentarón con un amplio conjunto de protocolos en diferentes entornos y 
configuraciones de red. Internet empezó a crecer debido a la extendida disponibilidad de software
y hardware que comenzó a soportar este sistema. El crecimiento del el tamaño y el alcance de Internet
y su uso cada vez mayor en aplicaciones comerciales ha despertado en investigadores, desarrolladores y
fabricantes la necesidad de desarrollar un \textit{framework} común de gestión para los productos
TCP / IP.

Para satisfacer estas necesidades, diferentes grupos empezaron a desarrollar conceptos de gestión
de redes que se pudieran aplicar a Internet y a sus tecnologías. Tres de estas tecnologías realizaron
suficientes progresos hacia finales de 1987. Estas tres tecnologías fueron el \textit{High-Level Entity Management System(HEMS)}, el \textit{Simple
Gateway Monitoring Protocol (SGMP)} y el \textit{Common Management Information Service/Protocol}.

Sin embargo, a corto plazo, Internet necesitaba alguna herramienta que solucionase
los problemas de gestión asociados con su rápido crecimiento. Dado el estado actual de la implementación
de SGMP y su simplicidad, el consenso general fue que este protocolo debe evolucionar a una especificación
más completa para poder realizar su despliegue de forma extendida. Poco después, \textbf{\textit{Simple 
Network Management Protocol (SNMP) }} sustituyo el protocolo SGMP por su facilidad de uso y versatilidad.

A principios del siglo XXI quedó de manifiesto que a pesar de su objetivo original, SNMP no se usaba para
tareas de configuración sino como una herramienta de monitorización de redes. En Junio de 2002 la La Junta
de Arquitectura de Internet (Internet Architecture Board o IAB) y miembros clave de la comunidad \gls{IETF}
se reunieron para evaluar la situación. Principalmente, los operadores de red utilizaban principalmente \glspl{CLI} propietarias para configurar sus
dispositivos. Además, muchos fabricantes ni siquiera proporcionaban la opción de configurar sus dispositivos
mediante SNMP. Estas \ac{CLI} tenían algunas características que gustaban a los operadores de red,
principalmente que usaban protocolos basados en texto a diferencia de \ac{SNMP} que usaba una codificación BER.
Aproximadamente al mismo tiempo, Juniper Neworks empezó a experimentar con sistemas de gestión basados en
\ac{XML}, hecho compartido con la \ac{IETF} que dio lugar a la creación del Grupo de Trabajo NETCONF cuyo
principal objetivo fue la creación de un protocolo de configuración de redes que se ajustase a las necesidades
de los operadores de red y fabricantes de dispositivos. La primera versión del protocolo se publicó en el
\rfclink{RFC4741} en Diciembre de 2006 con varias extensiones publicadas en los años siguientes. Una versión
mejorada del protocolo fue publicada en el \rfclink{RFC6241} en Junio de 2011.

En 2015 Google desarrollo el protocolo \ac{gRPC} de baja latencia basado el \ac{HTTP}/2.0. A continuación 
se desarrolla también el protocolo \ac{gNMI} proporcionando características parecidas a Netconf pero 
mejorando algunos aspectos. En primer lugar \ac{gNMI} usa Protobuf para serialización de datos dando resultado 
a mensajes de tamaño más reducido comparado con \ac{XML}. También soporta de forma nativa el streaming 
bidireccional de información muy útil para telemetría mientras que \ac{NETCONF} requiere usar las extensiones 
Yang PUSH. La última especificación de este protocolo es la versión 0.6.0 publicada el 30 de enero de 2018. 


% \subsection{Simple Network Mangement Protocol\label{sec:SNMP}}
% %%%%%%%%%%%%%%%%%%%%%%%%%%%%%%%%%%%%%%%%%%%%%%%%%%%%%%%%%%%%%%%%%%%%%%%%%%%%%%%%%%%%%%%%%%%%%%%%%%%%
\ac{SNMP} es un protocolo de capa de aplicación usado para la recolección y organización de
información de los dispositivos gestionados y la modificación de dicha información con el fin de 
modificar el comportamiento de dichos dispositivos. Entre los dispositivos que soportan SNMP 
encontramos módems, router, switches, servidores, estaciones de trabajo, etc.

La base de \ac{SNMP} es un simple conjunto de operaciones que permite a los administradores cambiar el
estado de dispositivos SNMP. Por ejemplo, se puede usar \ac{SNMP} para apagar una interfaz de un router o
comprobar la velocidad a la que una interfaz Ethernet está operando. 

\subsubsection{Gestores y Agentes}

En el mundo \ac{SNMP} existen dos tipos de entidades: gestores (\ac{SNMP} Managers) y agentes. 

%%%%%%%%%%%%%%%%%%%%%%%%%%%%%%%%%%%%%%%%%%%%%%%%%%%%%%%%%%%%%%%%%%%%%%%%%%%%%%%%%%%%%%%%%%%%%%%%%%%%
Un \textbf{SNMP Manager} también llamado Sistema de Gestión de Red (Network Management System o NMS)
es una entidad responsable de la comunicación con los agentes \ac{SNMP} disponible en la red. Típicamente
es un servidor que ejecuta uno o varios sistemas de gestión de red. Sus principales tareas son:

\begin{itemize}
    \item Consultar a los agentes
    \item Obtener las respuestas enviadas por los agentes
    \item Establecer o cambiar valores de variables en los agentes
    \item Recibir notificaciones eventos de los agentes de forma asíncrona
\end{itemize}

Los agentes SNMP con programas que recolectan información local de los dispositivos en los que están
instalador y la hacen disponible para los SNMP Managers. Las funciones de un agente SNMP son las
siguientes:

\begin{itemize}
    \item Recolectar información acerca de su entorno local
    \item Almacenar y devolver la información definida en sus bases de datos.
    \item Enviar señales a los gestores cuando ocurre un evento
    \item Actuar como un proxy para los nodos no gestionables mediante SNMP
\end{itemize}

En la Figura \ref{FIG:DIAGRAMA_COMUNICACIONES_SNMP} podemos observar como se organizan estas dos
entidades. 


\begin{figure}
    [Diagrama básico de comunicaciones SNMP]
    {FIG:DIAGRAMA_COMUNICACIONES_SNMP}
    {Diagrama básico de comunicaciones SNMP}
    \image{10cm}{}{snmp-components}
\end{figure}

%%%%%%%%%%%%%%%%%%%%%%%%%%%%%%%%%%%%%%%%%%%%%%%%%%%%%%%%%%%%%%%%%%%%%%%%%%%%%%%%%%%%%%%%%%%%%%%%%%%%
\subsubsection{\gls{MIB}}
La Base de Información Gestionada (\gls{MIB}) es un tipo de base de datos que contiene información
jerárquica, con estructura de árbol, de los parámetros gestionables de cada dispositivo SNMP. La
jerarquía MIB se organiza en distintos niveles que se asignan a distintas organizaciones. Los 
primeros niveles están asignados a organizaciones de normalización (ISO, CCITT, etc) mientras que 
los niveles más bajos están asignadas a organizaciones asociadas. En la Figura \ref{FIG:MIB_TREE} 
se puede observas esta forma de organización.

Existe un gran número de MIBs definidos por organizaciones como la \gls{IETF} así como entidades
privadas y fabricantes. La base de datos más común para la gestión de equipos en Internet es la base
MIB-II definida en el \rfclink{RFC1213} y ampliada con la aparición de las versiones 2 y 3 de SNMP. 
Este MIB es muy importante porque es obligatorio para todos los agentes SNMP de internet y contiene
información acerca del sistema, interfaces así como aspectos de IP (incluidas las tablas de
enrutamiento).

Todos los objetos tienen un identificador único denominado OID que permiten la identificación numérica
de cualquier nodo. Por ejemplo, sirviéndonos de la Figura \ref{FIG:MIB_TREE} podemos ver que el del
objeto \textit{iso.org.dod.internet.mgmt.mib-2.system.sysDescr} tiene el OID 1.3.6.1.2.1.1.1.

\begin{figure}
    [Árbol MIB (parcial)]
    {FIG:MIB_TREE}
    {Árbol MIB (parcial)}
    \image{10cm}{}{mib-oid-tree}
\end{figure}

%%%%%%%%%%%%%%%%%%%%%%%%%%%%%%%%%%%%%%%%%%%%%%%%%%%%%%%%%%%%%%%%%%%%%%%%%%%%%%%%%%%%%%%%%%%%%%%%%%%%
\subsubsection{Operaciones básicas de SNMP}

Las operaciones que pueden realizar dentro del protocolo SNMP son las siguientes
\cite{mauro2005essential}:

\begin{itemize}
    \item GET: La operación GET es una petición enviada por un gestor a un agente para obtener uno o
    más valores del agente. Figura \ref{FIG:SNMP_GET}

    \begin{figure}
        [SNMP Get Sequence]
        {FIG:SNMP_GET}
        {SNMP Get Sequence}
        \image{10cm}{}{snmp_get}
    \end{figure}
    
    \item GET NEXT: Esta operación es similar a GET. La principal diferencia es que GET NEXT obtiene
    el valor del siguiente OID del árbol MIB.
    \item GET BULK (SNMPv2 y v3): Operación usada para obtener grandes volúmenes de datos desde tablas
    MIB grandes.Figura \ref{FIG:SNMP_BULK}
    
    \begin{figure}
        [SNMP Get Bulk Sequence]
        {FIG:SNMP_BULK}
        {SNMP Get Bulk Sequence}
        \image{10cm}{}{snmp_get_bulk}
    \end{figure}

    \item SET: Esta operación es usada por los gestores para modificar o asignar un valor dentro de un
    agente.Figura \ref{FIG:SNMP_SET}
    
    \begin{figure}
        [SNMP Set Sequence]
        {FIG:SNMP_SET}
        {SNMP Set Sequence}
        \image{10cm}{}{snmp_set}
    \end{figure}
    
    \item TRAPS: Los TRAPS son señales enviadas por los agentes para notificar a los gestores de algún
    evento.Figura \ref{FIG:SNMP_TRAP}

    \begin{figure}
        [SNMP Trap Sequence]
        {FIG:SNMP_TRAP}
        {SNMP Trap Sequence}
        \image{10cm}{}{snmp_trap}
    \end{figure}
  
    \item INFORM: similar a una TRAP, pero a diferencia de este INFORM incluye la confirmación por
    parte del gestor \gls{SNMP} de la recepción del mensaje
    \item RESPONSE: este es el comando usado para devolver valores a los gestores SNMP.
\end{itemize}









% \subsection{Network Configuration Protocol\label{sec:NETCONF}}
% El protocolo \gls{NETCONF} es un protocolo de gestión de redes desarrollado y estandarizado por la \gls{IETF}. Fue desarrollado en el grupo de trabajo NETCONF publicado en diciembre de 2006 en el \rfclink{RFC4741} y posteriormente revisado en junio 2011 en el \rfclink{RFC6241}. 

\gls{NETCONF} proporciona los mecanispos para instalar, manipular y borrar la configuración de dispositivos de red. Las operaciones del protocolo \gls{NETCONF} se realizan mediante \glspl{RPC} y usan una codificación \gls{XML} tanto para los datos que constituyen la configuración como para los mensajes del propio protocolo. 

El protocolo se puede dividir conceptualmente en cuatro capas que se pueden ver en la Figura \ref{fig:capas_netconf}:
\begin{itemize}
    \item \textbf{Transporte}: La capa de transporte proporciona un canal de comunicación entre el cliente y el servidor.
    \item \textbf{Mensajes}: La capa de mensajes proporciona una forma simple, para enmarcar y codificar los mensajes del protocolo.
    \item \textbf{Operaciones}: La capa de operaciones define el conjunto de operaciones básicas y sus parámetros.
    \item \textbf{Contenido}: Define la organización y el modelado de los datos de configuración.
\end{itemize}

\begin{figure}
    \centering
    \includegraphics[scale=.75]{graphics/Capas_Netconf.pdf}
    \caption{Capas del protocolo \gls{NETCONF}}
    \label{fig:capas_netconf}
\end{figure}

\subsubsection{Transporte}

\gls{NETCONF} usa mensajes \glspl{RPC} para comunicarse. Un cliente manda una serie de solicitudes \gls{RPC} que provocan el el servidor responda con una serie de respuestas RPC. \gls{NETCONF} puede usar cualquier protocolo de transporte siempre que este proporcione la funcionalidad necesaria:

\begin{itemize}
    \item \textbf{Orientado a conexión}: \gls{NETCONF} requiere una conexión persistente entre pares.
    \item \textbf{Autenticación, Integridad y Confidencialidad}
\end{itemize}

Aunque existe soporte para muchos protocolos de transporte, los más usados son \gls{NETCONF} sobre \gls{SSH} \rfclink{RFC6242}\cite{RFC6242} y \gls{NETCONF} sobre \gls{TLS} con Autenticación mutua usando certificados X.509 \rfclink{RFC7589}\cite{RFC7589}.

\subsubsection{Mensajes}
La base del protocolo \gls{NETCONF} proporciona tres tipos de mensajes:

\begin{itemize}
    \item Solicitudes \gls{RPC} (mensajes <rpc>)
    \item Respuestas \gls{RPC} (mensajes <rpc-reply>)
    \item Notificaciones de eventos (mensajes <notification>)
\end{itemize}

Cada mensaje del protocolo \gls{NETCONF} es un documento \gls{XML} bien formado y codificado usando UTF-8. Las solicitudes y las respuestas se relacionan entre si mediante un atributo \enquote{message-id} de forma que el protocolo \gls{NETCONF} permite el \enquote{pipelining} de mensajes.

\subsubsection{Operaciones}

El protocolo \gls{NETCONF} proporciona un conjunto pequeño de operaciones de bajo nivel para gestionar la configuración de los dispositivos y recuperar información de estado. El protocolo base proporciona operaciones para recuperar, configurar, copiar y borrar elementos de configuración. Otras operaciones se pueden añadir en función de las capacidades anunciadas por el dispositivo. Las operaciones básicas son:

\begin{itemize}
    \item \textbf{get}: Recupera la configuración y la información de estado del dispositivo. A diferencia de \enquote{get-config} también puede recuperar información de estado.
    \item \textbf{get-config}: Recupera todo o parte de un datastore de configuración especifico. 
    \item \textbf{edit-config}: Carga todo o parte de una configuración a un datastore especifico.
    \item \textbf{copy-config}: Crea o reemplaza un datastore completo con los contenidos de otro datastore.
    \item \textbf{delete-config}: Elimina un datastore de configuración. El almacén de datos \enquote{running} no puede ser eliminado.
    \item \textbf{lock}: Permite a un cliente bloquear temporalmente un almacén de datos con el fin de realizar cambios sin interacciones por parte de otros clientes \gls{NETCONF}.
    \item \textbf{unlock}: Libera el bloqueo iniciado por una operación \enquote{lock} permitiendo de nuevo realizar modificaciones sobre el datastore afectado.
    \item \textbf{close-session}: Solicita la finalización de forma elegante de una sesión de \gls{NETCONF}.
    \item \textbf{kill-session}: Fuerza la finalización de una sesión \gls{NETCONF}
\end{itemize}

\subsubsection{Contenido}

La última revisión del protocolo \gls{NETCONF}, el \rfclink{RFC6241}\cite{RFC6241} no proporciona una definición para la capa de Contenido, quedando fuera del alcance de dicho documento. Los datos de configuración se definen mediante el lenguaje de modelado de datos YANG que se explicará en la Sección \ref{sec:yang_data_model}.






% \subsection{Google Network Management Interface\label{sec:gNMI}}
% \input{src/estado_del_arte/gnmi.tex}