    Tal como hemos especificado en el apartado anterior nuestro objetivo principal es crear una implementación de la extensión \enquote{Subscription to YANG Notifications for Datastore Updates} para un servidor NETCONF ya existente.

Los objetivos que se buscan alcanzar con el proyecto son los siguientes:

\begin{enumerate}[label=\textbf{O\arabic*} - , leftmargin=3\parindent]

    \item Desarrollar un \textbf{servidor} NETCONF capaz de:
        \begin{enumerate}[label=\textbf{O1.\arabic*} - , leftmargin=1\parindent]
            \item Gestionar notificaciones periódicas.
            \item Gestionar notificaciones de tipo \textit{on-change}.
            \item Soportar filtrado \textit{XPath} y \textit{subtree}
            \item Manejar correctamente los errores e informar a los clientes correctamente.
        \end{enumerate}    
    
    \item Desarrollar un \textbf{cliente} NETCONF que sea capaz de:
        \begin{enumerate}[label=\textbf{O2.\arabic*} - , leftmargin=1\parindent]
            \item Enviar peticiones de subscripción a servidores NETCONF que soporten la extensión de YANG-PUSH.
            \item Enviar modificaciones de subscripciones al servidor NETCONF.
            \item Guardar correctamente los datos enviados por los clientes para su posterior análisis mediante aplicaciones externas.
        \end{enumerate}    
    \item Desarrollar una interfaz para el cliente que permita a un administrador realizar conexiones a servidores, enviar peticiones de subscripción y visualizar los datos recibidos.
\end{enumerate}



