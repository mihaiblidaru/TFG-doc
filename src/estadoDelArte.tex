\chapter{Estado del Arte\label{sec:estado_del_arte}}

\section{Gestión de redes\label{sec:gestion_redes}}
El crecimiento de la complejidad de la estructura de las organizaciones en los últimos 40 años ha conllevado también aumento de la complejidad de sus redes de ordenadores. A medida que ampliamos y mejoramos nuestras redes, estas se vuelven más complejas y como resultado más difíciles de administrar. Si hace 40 años un administrador podría llevar a cabo su trabajo usando herramientas muy simples, con el tiempo estas han demostrado ser obsoletas e insuficientes y por este motivo se inició el desarrollo de las tecnologías de gestión de redes necesarias para igualar su nivel de sofisticación.

Durante su desarrollo, los investigadores, desarrolladores y usuarios del conjunto de protocolos DARPA/DoD TCP/IP han experimentado con un amplio conjunto de protocolos en diferentes entornos y configuraciones de red. La Internet empezó a crecer debido a la extendida disponibilidad de software y hardware que comenzó a soportar este sistema. El crecimiento del el tamaño y el alcance de Internet y cada vez su mayor uso en aplicaciones comerciales ha despertado en investigadores, desarrolladores y fabricantes la necesidad de desarrollar un \textit{framework} común de gestión para los productos TCP / IP.

Para satisfacer estas necesidades, diferentes esfuerzos empezaron a desarrollar conceptos de gestión de redes que se pudieran aplicar a Internet y a sus tecnologías. Tres de estas tecnologías realizaron suficientes progresos hacia finales de 1987 y quedo claro que la comunidad de desarrolladores tenía que tomar algunas decisiones para no acabar teniendo un conjunto de herramientas incompatibles entre sí. Estas tres tecnologías fueron el \textit{High-Level Entity Management System(HEMS)}, el \textit{Simple Gateway Monitoring Protocol (SGMP)} y el \textit{Common Management Information Service/Protocol}.

Sin embargo, a corto plazo, la Internet necesitaba desesperadamente alguna herramienta que solucionase los problemas de gestión asociados con su rápido crecimiento. Dado el estado actual de la implementación de SGMP y su simplicidad, el consenso general fue que este protocolo debe evolucionar a una especificación más completa para poder realizar su despliegue de forma extendida. Poco después, \textbf{\textit{Simple Network Management Protocol (SNMP) }} sustituyo el protocolo SGMP por su facilidad de uso y versatilidad.

A principios del siglo XXI quedó de manifiesto que a pesar de su objetivo original, SNMP no se usaba para tareas de configuración sino como una herramienta de monitorización de redes. En Junio de 2002 la La Junta de Arquitectura de Internet (Internet Architecture Board o IAB) y miembros clave de la comunidad \gls{IETF} se reunieron para evaluar la situación. Los resultados de esas reuniones están documentados en el RFC 3535. Principalmente, los operadores de red utilizaban principalmente \glspl{CLI} propietarias para configurar sus dispositivos. Además, muchos fabricantes ni siquiera proporcionaban la opción de configurar sus dispositivos mediante SNMP. Estas \gls{CLI} tenían algunas características que gustaban a los operadores de red, principalmente que usaban protocolos basados en texto a diferencia de SMNP que usaba una codificación BER. Aproximadamente al mismo tiempo, Juniper Neworks empezó a experimentar con sistemas de gestión basados en \gls{XML}, hecho compartido con la \gls{IETF} que dio  lugar a la creación del Grupo de Trabajo NETCONF cuyo principal objetivo fue la creación de un protocolo de configuración de redes que se ajustase a las necesidades de los operadores de red y fabricantes de dispositivos. La primera versión del protocolo se publicó en el \textit{RFC4741} en Diciembre de 2006 con varias extensiones publicadas en los años siguientes. Una versión mejorada del protocolo fue publicada en el \textit{RFC6241} en Junio de 2011.

En 2015 Google desarrollo el protocolo \gls{gRPC} de baja latencia basado el \gls{HTTP}/2.0. A continuación se desarrolla también el protocolo \gls{gNMI} proporcionando características parecidas a Netconf pero mejorando algunos aspectos. En primer lugar \gls{gNMI} usa Protobuf para serialización de datos dando resultado a mensajes de tamaño más reducido comparado con \gls{XML}. También soporta de forma nativa el streaming bidireccional de información muy útil para telemetría mientras que NETCONF requiere usar las extensiones Yang PUSH. La última especificación de este protocolo es la versión 0.6.0 publicada el 30 de enero de 2018.

\subsection{Simple Network Mangement Protocol\label{sec:SNMP}}
\subsubsection{Historia}
Simple Network Management Protocol (SNMP) es un conjunto de estándares para la gestión de redes, que incluye un protocolo especifico, una especificación de la estructura de los datos u un conjunto de objetos específicos. SNMP fue adoptado como estándar para las redes TCP/IP en el año 1989 

Es un protocolo de capa de aplicación usado para la recolección y organización de información de los dispositivos gestionados y la modificación de dicha información con el fin de modificar el comportamiento de dichos dispositivos. Entre los dispositivos que soportan SNMP encontramos modems, router, switches, servidores, estaciones de trabajo, etc.




\subsection{Network Configuration Protocol\label{sec:NETCONF}}

\subsection{Google Network Management Interface\label{sec:gNMI}}

\section{Yang Data Model}

\subsection{Introduccion al modelado YANG}

\subsection{OpenConfig modelos}

\section{Telemetria}

\subsection{Framework}

Resumen de
https://tools.ietf.org/id/draft-opsawg-ntf-00.html

\subsection{Network Configuration Notifications\label{sec:NETCONFNot}}

https://tools.ietf.org/html/rfc5277

\subsection{YANG Push Notifications\label{sec:YANGNot}}

https://datatracker.ietf.org/doc/rfc8641/
https://tools.ietf.org/id/draft-ietf-netconf-notification-capabilities-05.html
