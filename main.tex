% Definiciones y constantes de estilo
% Clase del documento
\documentclass[a4paper, 11pt, twoside, openright, titlepage]{book}

%
% Paquetes necesarios
%

% Símbolo del euro
\usepackage{eurosym}
% Codificación UTF8
\usepackage[utf8]{inputenc}
% Caracteres del español
\usepackage[spanish]{babel}
% Código, algoritmos, etc.
\usepackage{listings}
% Definición de colores
\usepackage{color}
% Extensión del paquete color
\usepackage[table,xcdraw]{xcolor}
% Márgenes
\usepackage{anysize}
% Cabecera y pie de página
\usepackage{fancyhdr}
% Estilo título capítulos
\usepackage{quotchap}
% Algoritmos (expresarlos mejor)
\usepackage{algorithmic}
% Títulos de secciones
\usepackage{titlesec}
% Fórmulas matemáticas
\usepackage[cmex10]{amsmath}
% Enumeraciones
\usepackage{enumerate}
% Páginas en blanco
\usepackage{emptypage}
% Separación entre cajas
\usepackage{float}
% Imágenes
\usepackage[pdftex]{graphicx}
% Mejora de las tablas
\usepackage{array}
% Mejora de los símbolos matemáticos
\usepackage{mdwmath}
% Separar figuras en subfiguras
\usepackage[caption=false,font=footnotesize]{subfig}
% Incluir pdfs externos
\usepackage{pdfpages}
% Mejoras sobre las cajas
\usepackage{fancybox}
% Apéndices
\usepackage{appendix}
% Marcadores (para el pdf)
\usepackage{bookmark}
% Estilo de enumeraciones
\usepackage{enumitem}
% Espacio entre líneas y párrafos
\usepackage{setspace}
% Glosario/Acrónimos
\usepackage[acronym]{glossaries}
% Fuentes
\usepackage[T1]{fontenc}
% Bibliografía
\usepackage[sorting=none,natbib=true,backend=biber,bibencoding=ascii]{biblatex}
% Fix biblatex+babel warning
\usepackage{csquotes}
% debug
\usepackage{etoolbox}
% Obtener fecha y tiempo actual
\usepackage{datetime2}
% Multiples imágenes por figura
\usepackage{graphicx}


% La versión del PDF generado será 1.7. Esto permite incluir gráficos en PDF cuya versión
% es inferior o igual a 1.7. Por defecto la versión es 1.5.
% El formato 1.7 se ha publicado en 2006 por lo que no debería tener ningún problema
\pdfminorversion=7

\DeclareBibliographyDriver{standard}{%
  \usebibmacro{bibindex}%
  \usebibmacro{begentry}%
  \usebibmacro{author}%
  \setunit{\labelnamepunct}\newblock
  \usebibmacro{title}%
  \newunit\newblock
  \printfield{number}%
  \setunit{\addspace}\newblock
  \printfield[parens]{type}%
  \newunit\newblock
  \usebibmacro{location+date}%
  \newunit\newblock
  \iftoggle{bbx:url}
    {\usebibmacro{url+urldate}}
    {}%
  \newunit\newblock
  \usebibmacro{addendum+pubstate}%
  \setunit{\bibpagerefpunct}\newblock
  \usebibmacro{pageref}%
  \newunit\newblock
  \usebibmacro{related}%
  \usebibmacro{finentry}}



% Enlaces
\hypersetup{hidelinks,pageanchor=true,colorlinks,citecolor=Fuchsia,urlcolor=black,linkcolor=Cerulean}

% Euro (€)
\DeclareUnicodeCharacter{20AC}{\euro}

% Inclusión de gráficos
\graphicspath{{./graphics/}}

% Texto referencias
\addto{\captionsspanish}{\renewcommand{\bibname}{Bibliografía}}

% Extensiones de gráficos
\DeclareGraphicsExtensions{.pdf,.jpeg,.jpg,.png,.gif}

% Definiciones de colores (para hidelinks)
\definecolor{LightCyan}{rgb}{0,0,0}
\definecolor{Cerulean}{rgb}{0,0,0}
\definecolor{Fuchsia}{rgb}{0,0,0}

\providetoggle{isDraft}
\settoggle{isDraft}{true}

% Keywords (español e inglés)
\def\keywordsEn{\vspace{.5em}
{\textbf{\textit{Key words ---}}\,\relax%
}}
\def\endkeywordsEn{\par}

\def\keywordsEs{\vspace{.5em}
{\textbf{\textit{Palabras clave ---}}\,\relax%
}}
\def\endkeywordsEs{\par}




% Abstract (español e inglés)
\def\abstractEs{\vspace{.5em}
{\textbf{\textit{Resumen ---}}\,\relax%
}}
\def\endabstractEs{\par}

\def\abstractEn{\vspace{.5em}
{\textbf{\textit{Abstract ---}}\,\relax%
}}
\def\endabstractEn{\par}

% Estilo páginas de capítulos
\fancypagestyle{plain}{
\fancyhf{}
\fancyfoot[CO]{\footnotesize\emph{\nombretrabajo}}
\fancyfoot[RO]{\thepage}
\renewcommand{\footrulewidth}{.6pt}
\renewcommand{\headrulewidth}{0.2pt}
}

% Estilo resto de páginas
\pagestyle{fancy}

% Estilo páginas impares
\fancyfoot[CO]{\footnotesize\emph{\nombretrabajo}}
\fancyfoot[RO]{\thepage}
\rhead[]{\leftmark}

% Estilo páginas pares
\fancyfoot[CE]{\emph{\pieparcen}}
\fancyfoot[LE]{\thepage}
\fancyfoot[RE]{\pieparizq}
\lhead[\leftmark]{}

% Guía del pie de página
\renewcommand{\footrulewidth}{.6pt}

% Nombre de los bloques de código
\renewcommand{\lstlistingname}{Código}

% Estilo de los lstlistings
\lstset{
    frame=tb,
    breaklines=true,
    postbreak=\raisebox{0ex}[0ex][0ex]{\ensuremath{\color{gray}\hookrightarrow\space}}
}

% Definiciones de funciones para los títulos
\newlength\salto
\setlength{\salto}{3.5ex plus 1ex minus .2ex}
\newlength\resalto
\setlength{\resalto}{2.3ex plus.2ex}

% Estilo de los acrónimos
\renewcommand{\acronymname}{Glosario}
\renewcommand{\glossaryname}{Glosario}
\pretolerance=2000
\tolerance=3000

% Texto índice de tablas
\addto\captionsspanish{
\def\tablename{Tabla}
\def\listtablename{\'Indice de tablas}
}

% Traducir appendix/appendices
\renewcommand\appendixtocname{Apéndices}
\renewcommand\appendixpagename{Apéndices}

% Comando code (lstlisting sin cambio de página)
\lstnewenvironment{code}[1][]%
  { \noindent\minipage{0.935\linewidth}\medskip
    \vspace{5mm}
    \lstset{basicstyle=\ttfamily\footnotesize,#1}}
  {\endminipage}
  
  
\newcommand{\rfclink}[1]{\href{https://tools.ietf.org/html/#1}{#1}}

\setlength{\headheight}{15.5pt}
% Definiciones de comandos
\newcommand{\nombreautor}{Mihai Blidaru}
\newcommand{\nombretutor}{Víctor López Álvarez}
\newcommand{\nombretrabajo}{NETCONF extensions to enable network telemetry}
\newcommand{\fecha}{\iftoggle{isDraft}{\DTMnow}{\today}}
\newcommand{\grado}{Grado en Ingeniería Informática}
% Descomentar si tu trabajo tiene un ponente
\newcommand{\nombreponente}{Jorge Enrique López de Vergara Méndez}
% Descomentar si tu trabajo está asociado a un grupo de investigación
% \newcommand{\grupoInvestigacion}{TODO: Grupo de investigación}
\newcommand{\departamento}{TODO: Departamento}
\newcommand{\facultad}{Escuela Politécnica Superior}
\newcommand{\universidad}{Universidad Autónoma de Madrid}
\newcommand{\pieparizq}{Netconf Telemetry}
\newcommand{\pieparcen}{Trabajo de Fin de Grado}
\newcommand{\logoizq}{Logo_EPS}
\newcommand{\logoder}{Logo_UAM_2020}
\newcommand{\correo}{mihai.blidaru@estudiante.uam.es}




% Glosario y acrónimos
\makeglossaries
% Acrónimos
% Los acrónimos solo aparecen si se usan en el texto con los comandos \gls...
% Si no se usa ningún acrónimo, la sección no aparece. Estas reglas se aplican también al glosario.

\newacronym{SNMP}{SNMP}{Simple Network Management Protocol}
\newacronym{IETF}{IETF}{Internet Engineering Task Force}
\newacronym{gRPC}{gRPC}{Google Remote Procedure Calls}
\newacronym{gNMI}{gNMI}{gRPC Network Management Interface}
\newacronym{XML}{XML}{Extensible Markup Language}
\newacronym[\glslongpluralkey={Interfaces de Línea de Comandos}]{CLI}{CLI}{Interfaz de Línea de Comandos}
\newacronym{HTTP}{HTTP}{Hypertext Transfer Protocol}
\newacronym{JSON}{JSON}{JavaScript Object Notation}
\newacronym{MIB}{MIB}{Management Information Base}
\newacronym{TFG}{TFG}{Trabajo de Fin de Grado}
\newacronym{RFC}{RFC}{Request for Comments}

% Glosario

\newglossaryentry{bitstream}{name={bitstream},description={En este contexto se refiere al binario que configura el Hardware de la FPGA}}


% Rerefencias
\bibliography{src/bibliografia}

% Inicio del documento
\begin{document}
% Elección del idioma (español)
\selectlanguage{spanish}

%
% Portada
%
\pagenumbering{gobble}
\include{portada}
\hypersetup{pageanchor=true}

% Estilo de párrafo de los capítulos
\setlength{\parskip}{0.75em}
\renewcommand{\baselinestretch}{1.00}
% Interlineado simple
\spacing{1}

%
% Agradecimientos
%
\pagenumbering{Roman}
\setcounter{page}{0}
\chapter*{Agradecimientos}

TODO: Agradecimientos

% Cita
\begin{flushright}
\textit{``It is the small everyday deeds of ordinary folk that keep\\ the darkness at bay. Small acts of kindness and love.''}
Gandalf
\end{flushright}
  

%
% Resumen
%
% Resumen en inglés
\chapter*{Abstract}

\begin{abstractEn}
TODO: Resumen en inglés, 250-500 palabras.


\end{abstractEn}

% Palabras clave en inglés
\begin{keywordsEn}
TODO: Palabras clave en inglés, separadas por coma.
\end{keywordsEn}

% Resumen en español
\chapter*{Resumen}

\begin{abstractEs}
TODO: Resumen en español, 250-500 palabras.


\end{abstractEs}

% Palabras clave en español
\begin{keywordsEs}
TODO: Palabras clave en español, separadas por coma.
\end{keywordsEs}


%
% Glosario
%
\printglossary[title=Glosario,toctitle=Glosario]
\printglossary[title=Acrónimos,toctitle=Acrónimos,type=\acronymtype]

% Estilo de párrafo de los índices
\setlength{\parskip}{1pt}
\renewcommand{\baselinestretch}{1}

%
% Tabla de contenidos
%
\tableofcontents
\listoftables
\listoffigures
\cleardoublepage

% Estilo de párrafo de los capítulos
\setlength{\parskip}{1em}
\renewcommand{\baselinestretch}{1}
% Interlineado simple
\spacing{1}
% Numeración contenido
\pagenumbering{arabic}
\setcounter{page}{1}
\raggedbottom
%
% Introducción
%
\chapter{Introducción}

    En este primer capítulo se explica en líneas generales el tema abordado en el proyecto. Se explica la motivación para la realización de este proyecto, nuestros objetivos y la estructura de este documento.

\section{Motivación}
    Esta memoria de \gls{TFG} tiene como propósito describir el objetivo, desarrollo y funcionamiento de las aplicaciones creadas para este proyecto.
    
    Hoy en día, miles de millones de dispositivos pueden conectarse a internet. Nuestra vida diaria también ha cambiado bastante con una gran cantidad de aplicaciones de IoT o aplicaciones móviles que están basadas en Internet(P.E etiquetas inteligentes, sensores de monitoreo de salud portátiles \enquote{wearables}, automóviles inteligentes, electrodomésticos conectados, etc). Sin embargo, la mayor cantidad de dispositivos conectados y la proliferación de servicios web y multimedia también suponen un gran impacto en la red que podría estar sujeta a mayores incidentes de red. 
    
    La \textbf{telemetría de red} describe cómo se puede recopilar información de varias fuentes de datos utilizando un conjunto de proceso de comunicación automatizados y transmitirlos a los equipos receptores para tareas de análisis. Estas tareas de análisis pueden incluir correlación de eventos, detección de anomalías, monitoreo de rendimiento, análisis de tendencias así como otros procesos relacionados. 
    
    La telemetría de red no supone solo un reto técnico sino también uno comercial. La ejecución exitosa de iniciativas de negocio depende de tener una visión de todas las piezas que componen un sistema. Elementos de almacenamiento, unidades de procesamiento y la infraestructura de transporte en una red son fundamentales para el éxito de las aplicaciones modernas y para una buena experiencia del usuario final. 
    
    \begin{figure}[H]
        \centering
        \includegraphics[scale=0.18]{graphics/Telemetry-Visibility-Analytics.png}
        \caption{Network telemetry, visibility and analytics}
        \label{fig:Telemetry_bussinesss}
    \end{figure}
    
    El objetivo de este \gls{TFG} es explorar una de las extensiones del protocolo NETCONF que permitiría el envío de notificaciones y datos de telemetría y desarrollar una implementación que demuestre las capacidades de esta extensión.
    
    El 17 de Abril de 2016 se publica el primer borrador denominado  \enquote{Subscribing to YANG datastore push updates} en el que se empieza a definir los mecanismos de subscripción de tipo \textit{push} para almacenes de datos (\textit{datastores}) que siguen un esquema YANG, que permiten a aplicaciones cliente solicitar notificaciones a un \textit{datastore} YANG que se envían desde un servidor NETCONF en base a una subscripción, sin necesidad de peticiones adicionales enviadas por el cliente\cite{draft-ietf-netconf-yang-push-00}.

    En Septiembre de 2019 tras 25 borradores intermedios se publica el \gls{RFC}8641 en el que se describen los mecanismos que permiten a un cliente solicitar un flujo continuo y personalizado de actualizaciones de un almacén de datos YANG. Estos mecanismos habilitan nuevas capacidades para el monitoreo remoto de la configuración y/o del estado operativo de un dispositivo\cite{RFC8641}.
    
    Junto con este \gls{RFC} la \gls{IETF} trabajó en paralelo desarrollando y completando otras especificaciones como podemos ver en la Figura \ref{fig:telemetry_history}. También podemos observar otros desarrollos por parte de otras organizaciones como OpenConfig y Gooogle (gNMI) que intentan definir su propio modelo para telemetría.
    
    \begin{figure}[H]
        \centering
        \includegraphics[scale=.5]{graphics/telemetry-history-2-1024x657.jpg}
        \caption{Model-driven Telemetry: Timeline}
        \label{fig:telemetry_history}
    \end{figure}
    
    Este trabajo se centrará en el RFC8641\cite{RFC8641}, que dado que ha sido publicado recientemente, y presenta una oportunidad para estar entre los primeros que implementen y demuestren las capacidades de la extensión definida en dicho \gls{RFC}.

\section{Objetivos}
    Tal como hemos especificado en el apartado anterior nuestro objetivo principal es crear una implementación de la extensión \enquote{Subscription to YANG Notifications for Datastore Updates} para un servidor NETCONF ya existente.

Los objetivos que se buscan alcanzar con el proyecto son los siguientes:

\begin{enumerate}[label=\textbf{O\arabic*} - , leftmargin=3\parindent]

    \item Desarrollar un \textbf{servidor} NETCONF capaz de:
        \begin{enumerate}[label=\textbf{O1.\arabic*} - , leftmargin=1\parindent]
            \item Gestionar notificaciones periódicas.
            \item Gestionar notificaciones de tipo \textit{on-change}.
            \item Soportar filtrado \textit{XPath} y \textit{subtree}
            \item Manejar correctamente los errores e informar a los clientes correctamente.
        \end{enumerate}    
    
    \item Desarrollar un \textbf{cliente} NETCONF que sea capaz de:
        \begin{enumerate}[label=\textbf{O2.\arabic*} - , leftmargin=1\parindent]
            \item Enviar peticiones de subscripción a servidores NETCONF que soporten la extensión de YANG-PUSH.
            \item Enviar modificaciones de subscripciones al servidor NETCONF.
            \item Guardar correctamente los datos enviados por los clientes para su posterior análisis mediante aplicaciones externas.
        \end{enumerate}    
    \item Desarrollar una interfaz para el cliente que permita a un administrador realizar conexiones a servidores, enviar peticiones de subscripción y visualizar los datos recibidos.
\end{enumerate}


\section{Estructura del documento}

Este memoria consta de los siguientes capítulos:
\begin{itemize}
    \item\textbf{Estado del Arte}: Dentro de este capítulo se explicarán cuales son las principales tecnologías y protocolos usados para la gestión de redes.
    \item\textbf{Diseño}: Dentro de este capítulo se explicaran las consideraciones y decisiones de diseño tomadas para cumplir con los objetivos de este proyecto.
    \item\textbf{Desarrollo}: En este capítulo se explicará el funcionamiento interno de la aplicación.
    \item\textbf{Resultados}: Dentro de este capitulo se mostrarán los resultados de las pruebas realizadas y se compararán con los requisitos del RFC8641\cite{RFC8641}.
    \item\textbf{Conclusiones y trabajo futuro}: En este capítulo se presentarán las conclusiones sobre el trabajo realizado y se explorarán posibles desarrollos futuros basados y relacionados con este proyecto. 
\end{itemize}

%
% Estado del arte
%
\chapter{Estado del Arte\label{sec:estado_del_arte}}

\section{Gestión de redes\label{sec:gestion_redes}}
\input{src/estado_del_arte/gestion_de_redes}

%%%%%%%%%%%%%%%%%%%%%%%%%%%%%%%%%%%%%%%%%%%%%%%%%%%%%%%%%%%%%%%%%
\section{Yang Data Model\label{sec:yang_data_model}}
\subsection{Introducción al modelado YANG}

\subsection{OpenConfig modelos}

%%%%%%%%%%%%%%%%%%%%%%%%%%%%%%%%%%%%%%%%%%%%%%%%%%%%%%%%%%%%%%%%%
\section{Telemetría}
\subsection{Framework}
% Resumen de https://tools.ietf.org/id/draft-opsawg-ntf-00.html

\subsubsection{Mecanismos de adquisición de datos}

En general, los datos de red se pueden obtener mediante subscripción (push) o consulta(poll). A su vez existen dos modelos de subscripción, Publish-Subscription y Subscription-Publish. En el perimero, una serie de datos predefinidos están publicados y varios subscriptores pueden solicitar esos datos. En el otro modo, un subscriptor comunica que datos quiere obtener y los dispositivos de red son los encargados de entregar esos datos cuando sean disponibles.

En cambio, un agente que realiza pooling espera una respuesta inmediata por parte de los dispositivos de red. 

Existen 4 tipos de datos que puede proporcional un dispositivo de red:
\begin{itemize}
    \item \textbf{Datos Simples}: datos que están siempre disponibles desde algún almacén de datos o sondas estáticas del dispositivo de red. Este tipo de datos puede ser especificado mediante un modelo YANG
    \item \textbf{Datos personalizados}: Estos datos necesitan ser sintetizados o procesados a partir de datos en bruto procedentes de uno o varios dispositivos de red.
    \item \textbf{Datos basados en eventos}: Los datos se generan condicionalmente en base a algún evento.
    \item \textbf{Datos de flujo}: Estos datos están siendo generados continuamente o de forma periódicamente.Los datos de flujo muestran el estado de un dispositivo real y requieren mucho ancho de banda y poder de procesamiento.
\end{itemize}

Como puede verse en la Figura \ref{fig:tipos_de_datos_framework} estos tipos de datos no nos excluyentes.

\begin{figure}
    \centering
    \includegraphics[scale=.75]{graphics/tipos_de_datos_framework}
    \caption{Relaciones entre los tipos de datos}
    \label{fig:tipos_de_datos_framework}
\end{figure}

En general las subscripciones trabajan con datos de flujo o datos basados en eventos mientras que el polling se usa más para datos simples o datos personalizados.


\subsubsection{Data Objects}
La telemetría se puede dividir también en cuatro módulos, en función del origen de los datos:

\begin{itemize}
    \item \textbf{Control Plane Telemetry}
    \item \textbf{Forwarding Plane Telemetry}
    \item \textbf{Management Plane Telemetry}
    \item \textbf{External Data and Event Telemery}
\end{itemize}


Las principales diferencias entre los 4 modulos los podemos ver en la Tabla \ref{tab:diferencias_modulos_data_objects}:
\begin{table}
    \centering
    \begin{tabular}{|p{2cm}|p{3cm}|p{3cm}|p{3cm}|p{3cm}|}
        \hline
        \textbf{Module} & \textbf{Control Plane} &\textbf{Management Plane} & \textbf{Forwarding Plane} & \textbf{External Data} \\\hline
        
        \textbf{Object} & control protocol \& signaling, RIB, ACL & config, operation state, MIB & flow \& packet QoS, trafic stat., buffer \& queue stat & terminal, social \& environmental\\\hline
        
        \textbf{Export Location }& main control CPU, linecard CPU or fwding chip & main control CPU & fwding chip or linecard CPU; main control CPU unlikely & various \\\hline
        
        \textbf{Model} & YANG, custom & MIB, syslog, YANG, custom & template, YANG custom & YANG\\\hline
        
        \textbf{Encoding} & GPB, JSON, XML, plain & GPB, JSON, XML & plain &  GPB, JSON, XML, plain\\\hline
        
        \textbf{Protocol} & gRPC, NETCONF, IPFIX, mirror & gRPC, NETCONF& IPFIX, mirror & gRPC\\\hline
        
        \textbf{Transport} & HTTP, TCP, UDP & HTTP, TCP & UDP & TCP, UDP \\\hline
        
        
        \end{tabular}

    \caption{Diferencias entre módulos del framework de telemetría de red}
    \label{tab:diferencias_modulos_data_objects}
\end{table}






\subsubsection{Evolución de la telemetría de red}
De la misma forma que las redes tienden a evolucionar hacia una forma de operar mucho más autónoma, la telemetría de red también se puede clasificar en varios niveles en función del nivel de autonomía:
\begin{enumerate}[label=Nivel \arabic*, leftmargin=4\parindent]
    \setcounter{enumi}{0}
    \item \textbf{Telemetría estática}: Los datos de telemetría se fijan en la fase de diseño.
    \item \textbf{Telemetría Dinámica}: Los datos de telemetría se pueden programar y configurar de forma dinámica en tiempo de ejecución.
    \item \textbf{Telemetría Interactiva}: Un operador de red puede alterar en tiempo real la telemetría. A este nivel algunas tareas se pueden automatizar aunque siempre se necesita un operador humano para tomar decisiones.
    \item \textbf{Circuito-Cerrado}: No hay operadores humanos. La red es inteligente y su sistema de control solicita datos de telemetría automáticamente, analiza los datos y modifica el funcionamiento de la red.
\end{enumerate}

La mayoría de las tecnologías existentes operan en el nivel 0 o 1 pero con la ayuda de un framework de telemetría bien definido se pueden crear las tecnologías para dar soporte al nivel 2 y dar los primeros pasos hacía el nivel 3.



\subsection{Network Configuration Notifications\label{sec:NETCONFNot}}

https://tools.ietf.org/html/rfc5277

\subsection{YANG Push Notifications\label{sec:YANGNot}}

https://datatracker.ietf.org/doc/rfc8641/
https://tools.ietf.org/id/draft-ietf-netconf-notification-capabilities-05.html

\clearpage
%
% Diseño
%
\chapter{Diseño\label{sec:disenho}}

\section{Base de datos}

Para poder implementar los servicios de notificaciones push es necesaria una forma de acceder a los
datos que se solicitan en las peticiones de tipo \textit{establish-subscription}. El sistema que 
proporcione estos datos debe cumplir varias condiciones:

\begin{enumerate}
    \item Primero, debe ser capaz de almacenar los datos siguiendo el esquema YANG correspondiente. 
    El lenguaje YANG define una estructura de datos en forma de árbol, definiendo la jerarquía entre
    objetos y pudiéndose codificar en diferentes lenguajes como pueden ser \gls{XML} o \gls{JSON}.
    Por ejemplo el siguiente modelo YANG ~\ref{lst:ejemplo-yang} se corresponde a los datos \gls{XML}
    listados en el Código \ref{lst:ejemplo-instancia-yang}:
    
    \lstinputlisting[label={lst:ejemplo-yang}, 
                        frame=single,
                        caption=Ejemplo de modelo YANG]
                        {src/code_snippets/ejemplo_yang_1.txt}
                        
    \lstinputlisting[label={lst:ejemplo-instancia-yang},
                        language=XML, 
                        frame=single,
                        caption=Una posible instancia del modelo YANG anterior]
                        {src/code_snippets/ejemplo_intancia_yang.xml}

    \item Debe ser compatible con, al menos, el lenguaje de programación utilizado, en este caso,
    Python 3.x, aunque es preferible que exista soporte para otros lenguajes.
    
    \item Debe proporcionar algún mecanismo de notificación de cambios de los datos para poder
    proporcionar notificaciones de tipo \textit{on-change} sin necesidad de hacer \textit{pooling}.
\end{enumerate}

En base a los criterios anteriores se han elegido tres bases de datos candidatas: MongoDB,
Prometheus y Redis. A continuación se van a evaluar cada una de ellas con el fin de elegir la base de datos más adecuada para el proyecto. En el Anexo \ref{appendix:bases_de_datos} se adjunta una tabla comparativa de los tres sistemas.

\subsection{MongoDB}
MongoDB es una base de datos multiplataforma orientada a documentos, clasificada como una base de 
datos NoSQL, que usa el lenguaje BSON, muy similar a JSON en vez del modelo tradicional de las 
bases de datos relacionales donde los datos se almacenan en filas. 

Como podemos ver en la figura~\ref{fig:Mongo_vs_RDBMS}, las tablas se corresponden a colecciones,
las filas se corresponden a documentos y las columnas se corresponden a campos de los documentos
pero con la ventaja de que MongoDB usa un modelo \textit{schemaless}, por tanto, los campos de los
documentos no son fijos. Dentro de una misma colección cada documento puede tener diferentes campos,
permitiendo la modificación de la estructura de un documento sin tener restricciones impuestas por
la colección (en un RDBMS todas las filas tienen las mismas columnas, es menos flexible). 

% TODO: añadir imágen  de objetos distintos en la misma colección emulando uno de los modelos yang.

\begin{figure}
    \centering
    \includegraphics[width=10cm]{graphics/MongoDB_vs_RMSBD}
    \caption{Equivalencias entre MongoDB y RDBMS}
    \label{fig:Mongo_vs_RDBMS}
\end{figure}

Como ya hemos visto en la introducción de este apéndice, al ser capaz de guardar datos con una
estructura \gls{JSON} es muy idoneo para este proyecto. Además es compatible con muchos lenguajes de
programación como C, CSharp, C++, Go, Java, Javascript, PHP y \textbf{Python}.
\subsection{Prometheus}
Prometheus \cite{prometheus} es una aplicación open-source usada para la monitorización de 
eventos y envío de alertas en tiempo real. Usa base de datos orientada a series temporales, 
dónde cada serie se identifica por un nombre, P.E \enquote{Used Bandwidth \%}, y un conjunto 
de pares clave-valor. Además, cuenta con un potente lenguaje de consultas llamado PromQL que
permite la manipulación de series temporales para la generación de gráficos, tablas y alertas.

Desgraciadamente, este sistema no cumple con varios de nuestros requisitos. En primer lugar,
está optimizado para series temporales mientras que para este proyecto se necesita guardar
instancias de modelos YANG. En segundo lugar, aunque Prometheus dispone de un sistema de alertas
llamado \textit{alertmanager} los tipos de notificaciones de proporciona no se corresponde con las 
necesidades del proyecto. Tal como se ha explicado al principio de esta sección, se necesita un 
mecanismo que detecte cambios en las instancias de los modelos YANG soportados y notificar el 
hilo/proceso que estuviese monitorizando un subárboles determinado de un modelo YANG. Sin embargo,
el \textit{alertmanager} lo que proporciona son alertas basadas en eventos (P.E Temperatura de la 
CPU > 90º durante más de 2 minutos) para administradores de sistemas, ofreciendo diferentes canales
como Slack, Telegram, Correo, SMS, etc. 


\begin{figure}
    \centering
    \includegraphics[width=15cm]{graphics/prometheus_architecture.png}
    \caption{Arquitectura de Prometheus}
    \label{fig:prometheus_estructura}
\end{figure}

Sin embargo, aunque no sea un sistema adecuado para almacenar los datos de un datastore YANG, se podría estudiar la utilización de Prometheus como sistema de visualización de series temporales que recibiría el cliente a través de notificaciones NETCONF, P.E. el cliente se podría subscribir para recibir la temperatura de la CPU cada 0.5 segundos y se podría utilizar Prometheus para almacenar esa serie temporal y usar sus herramientas de visualización para interpretar los datos 



\subsection{Redis} 
Redsis \cite{redsis_main_page} es un sistema de almacenamiento de datos en memoria de código abierto (licencia BSD) que se utiliza como base de datos, caché y como sistema de envío de mensajes. Admite estructuras de datos como cadenas, hashes, listas, conjuntos, conjuntos ordenados, mapas de bits, indices geoespaciales y flujos. Redis tiene replicación integrada, soporta scripting LUA, transacciones y distintos niveles de persistencia en disco. 

Redis guarda los datos en una estructura de tipo Hashtable como pares clave-valor soportando todos los
tipos de datos ya mencionados. Sin embargo, no permite la existencia de objetos anidados y por tanto no es capaz de representar un objeto YANG de forma completa (solo podría representar un solo nivel).
Además, no existe soporte para la mayoría de tipos de datos que podemos encontrar en las hojas de un modelo YANG como Integer16/32/64, Float y Double.

Redis tiene un sistema de alertas/notificaciones llamado Redis Keyspace Notifications que permite
a clientes subscribirse a canales para recibir notificaciones de eventos que afectan los datos de Redis. 

Sin embargo, debido a las limitaciones de este tipo de almacenamiento, considero que el sistema no es adecuado.

\begin{figure}
    \centering
    \includegraphics[width=15cm]{graphics/redis-data-structure-types.jpeg}
    \caption{Estructura de Datos de Redis}
    \label{fig:redis_estructura}
\end{figure}

\subsection{Decisión de diseño}





\section{Virtualizacion}
  \subsection{Maquinas Virtuales}
  \subsection{Docker}
  \subsection{Kubernetes}
\subsection{Decisión de diseño}

% \begin{figure}
%     \centering
%     \includegraphics{graphics/docker.png}
%     \caption{Caption}
%     \label{fig:my_label}
% \end{figure}

\clearpage
%
% Desarrollo
%


TODO: Desarrollo del proyecto
Test mod
\clearpage
%
% Resultados
%
\chapter{Resultados\label{sec:resultados}}

\section{Entorno de pruebas}

\section{Validación del protocolo YANG Push}

\subsection{Subscripciones periódicas}


\subsubsection{Period interval}

\paragraph{Flujo de trabajo}

diagrama de websequence

\paragraph{Intercambio de mensajes}


\subsubsection{Anchor-Time}

\paragraph{Flujo de trabajo}

diagrama de websequence

\paragraph{Intercambio de mensajes}



\subsection{Subscripciones sobre cambios}

dampening-period
change-type
sync-on-start (prioridad baja si da tiempo)

TODO: Pruebas y resultados

\subsection{Xpath}

\subsection{Performance}


\clearpage
%
% Conclusiones
%
\chapter{Conclusiones y trabajo futuro\label{sec:conclusiones}}

\section{Conclusiones}
TODO: Conclusiones sobre el trabajo realizado

\section{Trabajo futuro} 
\clearpage
%
% Página en blanco
%
\cleardoublepage

%
% Bibliografía
%
\printbibliography[heading=bibintoc]

% No expandir elementos para llenar toda la página
\raggedbottom

%
% Apéndices
%
\appendix
\cleardoublepage
\addappheadtotoc
\appendixpage

% Apéndices del TFG
\include{src/apendices}

% Fin del documento
\end{document}
