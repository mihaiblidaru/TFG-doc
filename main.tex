% arara: clean: {files: [main.aux, main.idx, main.ilg, main.ind, main.bbl, main.bcf, main.blg, main.run.xml, main.fdb_latexmk, main.fls, main.loe, main.lof, main.lol, main.lot, main.ltb, main.out, main.toc, main.upa, main.upb, main.acn, main.acr, main.alg, main.glg, main.glo, main.gls, main.glsdefs, main.idx,  main.ilg, main.xdy, main.loa, main.gnuploterrors , main.mw, main.fdb_latexmk ]}
% arara: pdflatex: {shell: yes}
% arara: makeglossaries
% arara: makeindex: {style: main.ist }
% arara: biber
% arara: pdflatex: {shell: yes}
% arara: pdflatex: {shell: yes}
% arara: clean: {files: [main.aux, main.idx, main.ilg, main.ind, main.bbl, main.bcf, main.blg, main.run.xml, main.fdb_latexmk, main.fls, main.loe, main.lof, main.lol, main.lot, main.ltb, main.out, main.toc, main.upa, main.upb, main.acn, main.acr, main.alg, main.glg, main.glo, main.gls, main.glsdefs, main.idx,  main.ilg, main.xdy, main.loa, main.gnuploterrors , main.mw, main.fdb_latexmk ]}

\documentclass[epsbased, copyright, final, covers, overleaf, printable, extendedindex, firstnumbered, tfg]{tfgtfmthesisuam}

% Comando para poner enlaces a RFCs solo con el nombre del RFC
% Ejemplo: \rfclink{RFC8461}. Si en el futuro la IETF cambia el formato
% de las URLs a RFC, este comando generará enlaces rotos
\newcommand{\rfclink}[1]{\href{https://tools.ietf.org/html/#1}{#1}}


\advisor{Victor Álvarez}
\levelin{Ingeniería Informática}
\title{Netconf YANG Push Extension}
%\speaker{Ponente}
%\subtitle{Si hace falta subtítulo}
\author{Mihai Blidaru}
\privateaddress{C\textbackslash\ Francisco Tomás y Valiente Nº 11}
\copyrightdate{3 de Noviembre de 2017}

%\dedication{A mi mujer y a mis hijos}
\famouscite{Lo peor es cuando has terminado un capítulo\\y la máquina de escribir no aplaude. \\[0.1em] \begin{flushright}Orson Welles\end{flushright}}
%\prefacefile{inicio/prefacio}
\ackfile{inicio/agradecimientos}
\resumenfile{inicio/resumen}
\abstractfile{inicio/abstract}

\keywords{Network, Telemetry, NETCONF, YANG, Notifications}
\palabrasclave{Redes, Telemetría, NETCONF, YANG, Notificaciones}

\coverdata
{
  Escuela Politécnica Superior \\
  Universidad Autónoma de Madrid \\
  C\textbackslash Francisco Tomás y Valiente nº 11
}

\bibliographyconfig{main}

\datadir{./data}
\graphicsdir{./graphics}
\logosdir{./imgOficiales}
\codesdir{./codes}

\begin{document}

% Acrónimos
% Los acrónimos solo aparecen si se usan en el texto con los comandos \gls...
% Si no se usa ningún acrónimo, la sección no aparece. Estas reglas se aplican también al glosario.

\newacronym{SNMP}{SNMP}{Simple Network Management Protocol}
\newacronym{IETF}{IETF}{Internet Engineering Task Force}
\newacronym{gRPC}{gRPC}{Google Remote Procedure Calls}
\newacronym{gNMI}{gNMI}{gRPC Network Management Interface}
\newacronym{XML}{XML}{Extensible Markup Language}
\newacronym[\glslongpluralkey={Interfaces de Línea de Comandos}]{CLI}{CLI}{Interfaz de Línea de Comandos}
\newacronym{HTTP}{HTTP}{Hypertext Transfer Protocol}
\newacronym{JSON}{JSON}{JavaScript Object Notation}
\newacronym{MIB}{MIB}{Management Information Base}
\newacronym{TFG}{TFG}{Trabajo de Fin de Grado}
\newacronym{RFC}{RFC}{Request for Comments}
\newacronym{NETCONF}{NETCONF}{Network Configuration Protocol}
\newacronym{RPC}{RPC}{Remote Procedure Call}
\newacronym{SSH}{SSH}{Secure Shell}
\newacronym{TLS}{TLS}{Transport Layer Security}
\newacronym{SSL}{SSL}{Secure Socket Layer}

% Glosario

%\newglossaryentry{bitstream}{name={bitstream},description={En este contexto se refiere al binario que configura el Hardware de la FPGA}}



\chapter{Introducción\label{CAP:INTRODUCION}}{introduccion/introduccion}
  \section{Motivación\label{SEC:MOTIVACION}}{introduccion/motivacion}
  \section{Objetivos\label{SEC:OBJETIVOS}}{introduccion/objetivos}
  \section{Estructura del documento\label{SEC:ESTRUCTURA_DEL_DOCUMENTO}}{introduccion/estructuraDelDocumento}

\chapter{Estado del Arte\label{CAP:ESTADO_DEL_ARTE}}{estadoDelArte/estadoDelArte}
  \section{Gestión de Redes\label{SEC:GESTION_DE_REDES}}{estadoDelArte/gestionDeRedes/gestionDeRedes}
    \subsection{Simple Network Mangement Protocol\label{SUBSEC:SNMP}}{estadoDelArte/gestionDeRedes/snmp}
    \subsection{Network Configuration Protocol\label{SUBSEC:NETCONF}}{estadoDelArte/gestionDeRedes/netconf}
    \subsection{Google Network Management Interface\label{SUBSEC:GNMI}}{estadoDelArte/gestionDeRedes/gnmi}
  \section{Yang Data Model\label{SEC:YANG_DATA_MODEL}}{estadoDelArte/yangDataModel}
  \section{Telemetría\label{SEC:TELEMETRIA}}{estadoDelArte/telemetria}

\chapter{Diseño\label{CAP:DISENHO}}{disenho/disenho}

\chapter{Desarrollo\label{CAP:DESARROLLO}}{desarrollo/desarrollo}

\chapter{Resultados\label{CAP:RESULTADOS}}{resultados/resultados}

\chapter{Conclusiones y trabajo futuro\label{CAP:CONCLUSIONES}}{conclusiones/conclusiones}

%% Aquí empiezan los apendices. Se cambia de numeración, etc.
\appendix

\chapter{Tabla comparativa bases de datos\label{appendix:bases_de_datos}}{apendices/apendiceA}

\end{document}
