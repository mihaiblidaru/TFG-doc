%% Authors: Rafael Leira Osuna (@ralequi)
%%          Jose Fernando Zazo (@jfzazo)
%%          Mario Daniel Ruiz Noguera (@mariodruiz)
%%          Sid (@JSidrach)
%%          Mihai Blidaru(yo)
%%          y otros que hayan colaborado para la construcción de esta plantilla
%%
%% Este fichero se debería importar con un \input y no con un \include. \input es lo
%% más parecido a un #include en C. Simplemente introduce el contenido de este fichero
%% en otro. \include hace más cosas
%%
%%
%% Esta plantilla parte de la base https://github.com/molins/tfg-plantilla a la que he
%% añadido las modificaciones que he considerado convenientes a lo largo del desarrollo
%% de mi propio TFG.
%%
%% Date: 05/07/2020
%%
%% Este fichero contiene toda la configuración global del documento latex. Esto incluye
%% la clase del documento, la importación de paquetes, la definición de nuevos comandos
%% y todos aquellos ajustes que se pueden realizar desde un contexto global.

% Clase del documento

%% a4paper   -> Tamaño A4.
%% 10pt      -> Tamaño de la fuente.
%% twoside   -> Genera dos tipos de páginas, izquierda y derecha.
%% openright -> Cada capítulo siempre empieza en una página a la derecha imaginando que el documento generado fuese un libro.
%% titlepage -> Después de un \maketitle automáticamente empieza una nueva página
%% book      -> Es una clase de documentos que permite tener capítulos
\documentclass[a4paper, 10pt, twoside, openright, titlepage]{book}

%
% Paquetes necesarios
%

% Símbolo del euro
\usepackage{eurosym}

% Codificación UTF8
\usepackage[utf8]{inputenc}

% Caracteres del español
\usepackage[spanish]{babel}

% Código, algoritmos, etc.
\usepackage{listings}

% Definición de colores
\usepackage{color}

% Extensión del paquete color
\usepackage[table,xcdraw]{xcolor}

% Márgenes
\usepackage{anysize}

% Cabecera y pie de página
\usepackage{fancyhdr}

% Estilo título capítulos
\usepackage{quotchap}

% Algoritmos (expresarlos mejor)
\usepackage{algorithmic}

% Títulos de secciones
\usepackage{titlesec}

% Fórmulas matemáticas
\usepackage[cmex10]{amsmath}

% Enumeraciones
\usepackage{enumerate}

% Páginas en blanco
\usepackage{emptypage}

% Separación entre cajas
\usepackage{float}

% Imágenes.
\usepackage[pdftex]{graphicx}

% Mejora de las tablas
\usepackage{array}

% Mejora de los símbolos matemáticos
\usepackage{mdwmath}

% Separar figuras en subfiguras
\usepackage[caption=false,font=footnotesize]{subfig}

% Incluir pdfs externos
\usepackage{pdfpages}

% Mejoras sobre las cajas
\usepackage{fancybox}

% Apéndices
\usepackage{appendix}

% Marcadores (para el pdf)
\usepackage{bookmark}

% Estilo de enumeraciones
\usepackage{enumitem}

% Espacio entre líneas y párrafos
\usepackage{setspace}

% Glosario/Acrónimos
\usepackage[acronym]{glossaries}

% Fuentes
\usepackage[T1]{fontenc}

% Bibliografía
\usepackage[sorting=none,natbib=true,backend=biber,bibencoding=ascii]{biblatex}

% Fix biblatex+babel warning
\usepackage{csquotes}

% Definiciones de comandos
\newcommand{\nombreautor}{Mihai Blidaru}
\newcommand{\nombretutor}{Víctor López Álvarez}
\newcommand{\nombretrabajo}{NETCONF extensions to enable network telemetry}
\newcommand{\fecha}{\today}
\newcommand{\grado}{Grado en Ingeniería Informática}
% Descomentar si tu trabajo tiene un ponente
\newcommand{\nombreponente}{Jorge Enrique López de Vergara Méndez}
% Descomentar si tu trabajo está asociado a un grupo de investigación
% \newcommand{\grupoInvestigacion}{TODO: Grupo de investigación}
\newcommand{\departamento}{TODO: Departamento}
\newcommand{\facultad}{Escuela Politécnica Superior}
\newcommand{\universidad}{Universidad Autónoma de Madrid}
\newcommand{\pieparizq}{Netconf Telemetry}
\newcommand{\pieparcen}{Trabajo de Fin de Grado}
\newcommand{\logoizq}{Logo_EPS}
\newcommand{\logoder}{Logo_UAM_2020}
\newcommand{\correo}{mihai.blidaru@estudiante.uam.es}

%% La versión del PDF generado será 1.7. Esto permite incluir gráficos en PDF cuya versión es inferior o igual a 1.7. Por defecto la versión es 1.5. El formato 1.7 se ha publicado en 2006 por lo que no debería tener ningún problema
\pdfminorversion=7

% Metadatos para el PDF generado. Si alguien abre este PDF, el título que le muestra el 
% lector es el título del trabajo, no algo autogenerado o algo sin sentido
\hypersetup{
    pdftitle={\nombretrabajo},
    pdfauthor={\nombreautor},
    pdfcreator={Overleaf}
}

% Enlaces
\hypersetup{hidelinks,
        pageanchor=true,
        colorlinks,
        citecolor=black,
        urlcolor=black,
        linkcolor=black}

% Euro (€)
\DeclareUnicodeCharacter{20AC}{\euro}

% Inclusión de gráficos
\graphicspath{{./graphics/}}

% Extensiones de gráficos
\DeclareGraphicsExtensions{.pdf,.jpeg,.jpg,.png,.gif}

% Texto referencias
\addto{\captionsspanish}{\renewcommand{\bibname}{Bibliografía}}

% Keywords (español e inglés)
\def\keywordsEn{\vspace{.5em}
{\textbf{\textit{Key words ---}}\,\relax}}
\def\endkeywordsEn{\par}

\def\keywordsEs{\vspace{.5em}
{\textbf{\textit{Palabras clave ---}}\,\relax}}
\def\endkeywordsEs{\par}

% Abstract (español e inglés)
\def\abstractEs{\vspace{.5em}
{\textbf{\textit{Resumen ---}}\,\relax}}
\def\endabstractEs{\par}

\def\abstractEn{\vspace{.5em}
{\textbf{\textit{Abstract ---}}\,\relax}}
\def\endabstractEn{\par}

% Estilo páginas de capítulos
\fancypagestyle{plain}{
    \fancyhf{}
    \fancyfoot[CO]{\footnotesize\emph{\nombretrabajo}}
    \fancyfoot[RO]{\thepage}
    \renewcommand{\footrulewidth}{.6pt}
    \renewcommand{\headrulewidth}{0.2pt}
}

% Estilo resto de páginas
\pagestyle{fancy}

% Estilo páginas impares
% O = odd (páginas impares)
% R = right
\fancyfoot[CO]{\footnotesize\emph{\nombretrabajo}}
\fancyfoot[RO]{\thepage}
\rhead[]{\leftmark}

% Estilo páginas pares
% E = even (páginas pares)
% C = center
% L = left
% R = right
\fancyfoot[CE]{\emph{\pieparcen}}
\fancyfoot[LE]{\thepage}
\fancyfoot[RE]{\pieparizq}
\lhead[\leftmark]{}

% Guía del pie de página
\renewcommand{\footrulewidth}{.6pt}

% Nombre de los bloques de código
\renewcommand{\lstlistingname}{Código}

% Estilo de los lstlistings
\lstset{
    frame=tb,
    breaklines=true,
    postbreak=\raisebox{0ex}[0ex][0ex]{\ensuremath{\color{gray}\hookrightarrow\space}}
}

% Definiciones de funciones para los títulos
\newlength\salto
\setlength{\salto}{3.5ex plus 1ex minus .2ex}
\newlength\resalto
\setlength{\resalto}{2.3ex plus.2ex}

% Estilo de los acrónimos
\renewcommand{\acronymname}{Glosario}
\renewcommand{\glossaryname}{Glosario}
\pretolerance=2000
\tolerance=3000

% Texto índice de tablas
\addto\captionsspanish{
    \def\tablename{Tabla}
    \def\listtablename{\'Indice de tablas}}

% Traducir appendix/appendices
\renewcommand\appendixtocname{Apéndices}
\renewcommand\appendixpagename{Apéndices}

% Comando code (lstlisting sin cambio de página)
\lstnewenvironment{code}[1][]
  { \noindent\minipage{0.935\linewidth}\medskip
    \vspace{5mm}
    \lstset{basicstyle=\ttfamily\footnotesize,#1}}
  {\endminipage}
  
% Comando para poner enlaces a RFCs solo con el nombre del RFC
% Ejemplo: \rfclink{RFC8461}. Si en el futuro la IETF cambia el formato
% de las URLs a RFC, este comando generará enlaces rotos
\newcommand{\rfclink}[1]{\href{https://tools.ietf.org/html/#1}{#1}}

